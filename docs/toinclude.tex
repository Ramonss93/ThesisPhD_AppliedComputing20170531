% \usepackage{etex}
\usepackage{makeidx}
\usepackage{graphicx}
\usepackage{multirow}
%\usepackage{caption}
\usepackage{changes}
\usepackage{algpseudocode}
\usepackage{algorithm}
\usepackage{parskip}
\usepackage{rotating}
\usepackage[graphicx]{realboxes}
\usepackage{xcolor}
\usepackage{color}
\usepackage{titlesec}
\usepackage{mathtools}
\usepackage[outline]{contour}
\usepackage{adjustbox}
\usepackage{enumitem}
\usepackage[final]{listings}
\usepackage{array}
\usepackage{forest}
\usepackage[skins]{tcolorbox}
\usepackage{enumitem}
\usepackage{longtable}
\usepackage{pdflscape}
\usepackage{pgfplotstable}
\usepackage{pgfplots}
\usepackage{pgfgantt}
\usepackage{tikzscale}
\usepackage{breakcites}
\usepackage{amsmath}
\usepackage{booktabs}
\usepackage{dirtytalk}
\usepackage{makecell}
% \usepackage{subcaption}
\usepackage{smartdiagram}
\usepackage{hyperref}
\usepackage{cleveref}
\usepackage{calligra}
\usepackage[T1]{fontenc}
\usepackage{acronym}
\usepackage{epigraph}
\usepackage{morewrites}
\usepackage{pdfrender}

\newcommand{\subfigureautorefname}{Figure}

\renewcommand{\epigraphsize}{\small}
\setlength{\epigraphwidth}{0.6\textwidth}
\renewcommand{\textflush}{flushright}
\renewcommand{\sourceflush}{flushright}
% A useful addition
\newcommand{\epitextfont}{\itshape}
\newcommand{\episourcefont}{\scshape}

\makeatletter
\newsavebox{\epi@textbox}
\newsavebox{\epi@sourcebox}
\newlength\epi@finalwidth
\renewcommand{\epigraph}[2]{%
  \vspace{\beforeepigraphskip}
  {\epigraphsize\begin{\epigraphflush}
   \epi@finalwidth=\z@
   \sbox\epi@textbox{%
     \varwidth{\epigraphwidth}
     \begin{\textflush}\epitextfont#1\end{\textflush}
     \endvarwidth
   }%
   \epi@finalwidth=\wd\epi@textbox
   \sbox\epi@sourcebox{%
     \varwidth{\epigraphwidth}
     \begin{\sourceflush}\episourcefont#2\end{\sourceflush}%
     \endvarwidth
   }%
   \ifdim\wd\epi@sourcebox>\epi@finalwidth 
     \epi@finalwidth=\wd\epi@sourcebox
   \fi
   \leavevmode\vbox{
     \hb@xt@\epi@finalwidth{\hfil\box\epi@textbox}
     \vskip1.75ex
     \hrule height \epigraphrule
     \vskip.75ex
     \hb@xt@\epi@finalwidth{\hfil\box\epi@sourcebox}
   }%
   \end{\epigraphflush}
   \vspace{\afterepigraphskip}}}
\makeatother

\makeatletter 
\renewcommand\thealgorithm{\thechapter.\arabic{algorithm}} 
\@addtoreset{algorithm}{chapter} 
\makeatother

\pgfplotsset{compat=newest}
% \usepackage{definition}

\renewcommand{\floatpagefraction}{.9}%

\setlist[itemize]{topsep=0pt,before=\leavevmode\vspace{-1.5em}}
\setlist[description]{style=nextline}

\newcommand*{\tabref}[1]{\tablename~\ref{#1}}
\newcommand*{\figref}[1]{\figurename~\ref{#1}}
\newcommand*{\secref}[1]{Section~\ref{#1} \:}
\newcommand*{\algoref}[1]{Algorithm~\ref{#1}}

\newcolumntype{L}[1]{>{\raggedright\let\newline\\\arraybackslash\hspace{0pt}}p{#1}}
\newcolumntype{C}[1]{>{\centering\let\newline\\\arraybackslash\hspace{0pt}}p{#1}}
\newcolumntype{R}[1]{>{\raggedleft\let\newline\\\arraybackslash\hspace{0pt}}p{#1}}

\newcommand{\algorithmautorefname}{Algorithm}

\setcounter{secnumdepth}{4}

\titleformat{\paragraph}
{\normalfont\normalsize\bfseries}{\theparagraph}{1em}{}
\titlespacing*{\paragraph}
{0pt}{3.25ex plus 1ex minus .2ex}{1.5ex plus .2ex}

\newganttchartelement*{mymilestone}{
mymilestone/.style={
shape=isosceles triangle,
inner sep=0pt,
draw=cyan,
top color=white,
bottom color=cyan!50
},
mymilestone incomplete/.style={
/pgfgantt/mymilestone,
draw=yellow,
bottom color=yellow!50
},
mymilestone label font=\slshape,
mymilestone left shift=0pt,
mymilestone right shift=0pt
}

\newgantttimeslotformat{stardate}{%
\def\decomposestardate##1.##2\relax{%
\def\stardateyear{##1}\def\stardateday{##2}%
}%
\decomposestardate#1\relax%
\pgfcalendardatetojulian{\stardateyear-01-01}{#2}%
\advance#2 by-1\relax%
\advance#2 by\stardateday\relax%
}

\definecolor{bblue}{rgb}{0,0,.4}
\captionsetup{font=footnotesize}

\newcommand{\specialcell}[2][c]{%
  \begin{tabular}[#1]{@{}c@{}}#2\end{tabular}}

\algnewcommand\algorithmicswitch{\textbf{switch}}
\algnewcommand\algorithmiccase{\textbf{case}}
\algnewcommand\algorithmicassert{\texttt{assert}}
\algnewcommand\Assert[1]{\State \algorithmicassert(#1)}%

\algdef{SE}[SWITCH]{Switch}{EndSwitch}[1]{\algorithmicswitch\ #1\ \algorithmicdo}{\algorithmicend\ \algorithmicswitch}
\algdef{SE}[CASE]{Case}{EndCase}[1]{\algorithmiccase\ #1}{\algorithmicend\ \algorithmiccase}

\algblockdefx[Default]{Default}{EndDefault}[1][]{\textbf{default} #1}{end default}

\algtext*{EndSwitch}
\algtext*{EndCase}
\algtext*{EndWhile}
\algtext*{EndIf}
\algtext*{EndFor}
\algtext*{EndFunction}
\algtext*{EndDefault}

\bgroup
\def\arraystretch{1.5}

\usetikzlibrary{shapes}
\usetikzlibrary{plotmarks}
\usetikzlibrary{decorations.pathmorphing,patterns}
\usetikzlibrary{calc,patterns,decorations.markings}
\usetikzlibrary{shapes,arrows,calc,intersections}
\usetikzlibrary{decorations.markings}
\usetikzlibrary{decorations.pathreplacing}
\usetikzlibrary{shapes}
\usetikzlibrary{plotmarks}
\usetikzlibrary{decorations.pathmorphing,patterns}
\usetikzlibrary{calc,patterns,decorations.markings}
\usetikzlibrary{shapes,arrows,calc,intersections}
\usetikzlibrary{decorations.markings}
\usetikzlibrary{shapes}
\usetikzlibrary{plotmarks}
\usetikzlibrary{decorations.pathmorphing,patterns}
\usetikzlibrary{calc,patterns,decorations.markings}
\usetikzlibrary{shapes,arrows,calc,intersections}
\usetikzlibrary{decorations.markings}
\usetikzlibrary{mindmap,trees}
\usetikzlibrary{shadows}
\usetikzlibrary{shapes,arrows,calc,intersections}
\usetikzlibrary{decorations.markings}
\usetikzlibrary{decorations.pathreplacing} 
\usetikzlibrary{angles,quotes}
\usetikzlibrary{decorations.markings} 
\usetikzlibrary{shapes.geometric}
\usetikzlibrary{positioning}
\usetikzlibrary{calc,patterns,decorations.pathmorphing,decorations.markings}
\usetikzlibrary{patterns}
\usetikzlibrary{pgfplots.groupplots}
\usetikzlibrary{pgfplots.statistics}
\usetikzlibrary{arrows.meta}


\tikzstyle{platform}=[fill,pattern=north east lines,draw=none,minimum width=1cm,minimum height=0.3cm]

\makeatletter
\def\underscoreusetikzlibrary{\pgfutil@ifnextchar[{\underscoreuse@tikzlibrary}{\underscoreuse@@tikzlibrary}}%}
\def\underscoreuse@tikzlibrary[#1]{\underscoreuse@@tikzlibrary{#1}}
\def\underscoreuse@@tikzlibrary#1{%
  \edef\pgf@list{#1}%
  \pgfutil@for\pgf@temp:=\pgf@list\do{%
	\expandafter\pgfkeys@spdef\expandafter\pgf@temp\expandafter{\pgf@temp}%
	\ifx\pgf@temp\pgfutil@empty
	\else
		\expandafter\ifx\csname tikz@library@\pgf@temp @loaded\endcsname\relax%
		  \expandafter\global\expandafter\let\csname tikz@library@\pgf@temp @loaded\endcsname=\pgfutil@empty%
		  \expandafter\edef\csname tikz@library@#1@atcode\endcsname{\the\catcode`\@}
		  \expandafter\edef\csname tikz@library@#1@barcode\endcsname{\the\catcode`\|}
		  \catcode`\@=11
		  \catcode`\|=12
		  \input tikzlibrary\pgf@temp_code.tex
		  \catcode`\@=\csname tikz@library@#1@atcode\endcsname
		  \catcode`\|=\csname tikz@library@#1@barcode\endcsname
		\fi%
	\fi
  }%
}
\makeatother
\underscoreusetikzlibrary{mec}

\contourlength{1pt}

\renewcommand{\arraystretch}{0.9}

\tikzstyle{startstop} = [rectangle, rounded corners, minimum width=3cm, minimum height=1cm,text centered, draw=black]
\tikzstyle{io} = [trapezium, trapezium left angle=70, trapezium right angle=110, minimum width=2cm, minimum height=1cm, text centered, draw=black]
\tikzstyle{process} = [rectangle, minimum width=3cm, minimum height=1cm, text centered, draw=black]
\tikzstyle{decision} = [diamond, minimum width=3cm, minimum height=1cm, text centered, draw=black, aspect=2.5]
\tikzstyle{arrow} = [thick,->,>=stealth]
\definecolor{myyellow}{RGB}{254,241,24}
\definecolor{myorange}{RGB}{234,125,1}

\def\proton(#1,#2){%
    \fill[ball color=yellow] (#1,#2) circle (10pt);
}
\def\neutron(#1,#2){%
    \fill[ball color=blue] (#1:#2) circle (10pt);
}
\def\neutronrejeited(#1,#2){%
    \fill[ball color=blue,fill opacity=.4,draw,densely dotted] (#1:#2) circle (10pt);
}
\def\electron(#1,#2){%
    \fill[ball color=green!30] (#1:#2) circle (10pt);
}

\tikzset{->-/.style={decoration={
  markings,
  mark=at position #1 with {\arrow{latex}}},postaction={decorate}}
}

\theoremstyle{definition}
\newtheorem{definition}{Definition}[chapter]
% \newenvironment{definition}
%   {\pushQED{\qed}\renewcommand{\qedsymbol}{$\triangle$}\examplex}
%   {\popQED\endexamplex}
\providecommand*\definitionautorefname{Definition}

\def\myrad{2cm}
\def\myang{30}

\pgfdeclarepatternformonly{south west lines}{\pgfqpoint{-0pt}{-0pt}}{\pgfqpoint{3pt}{3pt}}{\pgfqpoint{3pt}{3pt}}{
        \pgfsetlinewidth{0.4pt}
        \pgfpathmoveto{\pgfqpoint{0pt}{0pt}}
        \pgfpathlineto{\pgfqpoint{3pt}{3pt}}
        \pgfpathmoveto{\pgfqpoint{2.8pt}{-.2pt}}
        \pgfpathlineto{\pgfqpoint{3.2pt}{.2pt}}
        \pgfpathmoveto{\pgfqpoint{-.2pt}{2.8pt}}
        \pgfpathlineto{\pgfqpoint{.2pt}{3.2pt}}
        \pgfusepath{stroke}}
        
\newcommand{\R}{\mathbb{R}}


\definecolor{armygreen}{rgb}{0.29, 0.33, 0.13}
\definecolor{oucrimsonred}{rgb}{0.6, 0.0, 0.0}
\definecolor{yaleblue}{rgb}{0.06, 0.3, 0.57}
\definecolor{sandstorm}{rgb}{0.93, 0.84, 0.25}
\definecolor{vegasgold}{rgb}{0.77, 0.7, 0.35}

\pgfplotsset{compat=1.11,
    /pgfplots/ybar legend/.style={
    /pgfplots/legend image code/.code={%
       \draw[##1,/tikz/.cd,yshift=-0.25em]
        (0cm,0cm) rectangle (3pt,0.8em);},
   },
}

\makeatletter

% define a macro \Autoref to allow multiple references to be passed to \autoref
\newcommand\Autoref[1]{\@first@ref#1,@}
\def\@throw@dot#1.#2@{#1}% discard everything after the dot
\def\@set@refname#1{%    % set \@refname to autoefname+s using \getrefbykeydefault
    \edef\@tmp{\getrefbykeydefault{#1}{anchor}{}}%
    \def\@refname{\@nameuse{\expandafter\@throw@dot\@tmp.@autorefname}s}%
}
\def\@first@ref#1,#2{%
  \ifx#2@\autoref{#1}\let\@nextref\@gobble% only one ref, revert to normal \autoref
  \else%
    \@set@refname{#1}%  set \@refname to autoref name
    \@refname~\ref{#1}% add autoefname and first reference
    \let\@nextref\@next@ref% push processing to \@next@ref
  \fi%
  \@nextref#2%
}
\def\@next@ref#1,#2{%
   \ifx#2@ and~\ref{#1}\let\@nextref\@gobble% at end: print and+\ref and stop
   \else, \ref{#1}% print  ,+\ref and continue
   \fi%
   \@nextref#2%
}

\makeatother

\newcommand*{\fullref}[1]{\hyperref[{#1}]{\Cref*{#1} \nameref*{#1}}} % One single link

\makeatletter
\AtBeginDocument{%
  \let\l@algorithm\l@figure%
  \let\listofalgorithms\listoffigures% Copy \listoffigures
  \let\@cftmakeloatitle\@cftmakeloftitle% Copy LoF title
  % Update LoA-related macros
  \patchcmd{\listofalgorithms}{\@cftmakeloftitle}{\@cftmakeloatitle}{}{}%
  \patchcmd{\listofalgorithms}{\@starttoc{lof}}{\@starttoc{loa}}{}{}%
  \patchcmd{\@cftmakeloatitle}{\listfigurename}{\listalgorithmname}{}{}%
  % Add per-chapter LoA space (similar to LoF)
  \patchcmd{\@chapter}{\addtocontents}{%
    \addtocontents{loa}{\protect\addvspace{10\p@}}%
    \addtocontents}{}{}%
}

\makeatother