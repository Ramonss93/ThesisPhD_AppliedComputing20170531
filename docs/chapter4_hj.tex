\section{Hooke-Jeeves Pattern Search Method}
\label{sec:hj}

The direct search method proposed by \citeonline{hooke1961direct} consists of the repeated application of exploratory moves around a base point $s$ defined in a $n$-dimensional space. If these steps are successful, they are followed by pattern moves.

Hooke-Jeeves (HJ) method has been used in some hybrid algorithms, such as HJPCA \cite{RiosCoelho2010843}, HJMA \cite{Moser2009AHB}, PSO/HJ \cite{kampf2010comparison}, GAHJ \cite{long2014hybrid}, AS+HJ \cite{Braun2015}, qG-HJ \cite{HernandezTorres2015}, and MPCA-HJ \cite{HernandezTorres2015b}.

\Autoref{fig:hj,fig:hje}, and \Autoref{alg:hj, alg:hjexploratory} show the flowchart and the pseudo-code of the HJ method, respectively.

\begin{figure}[H]
\caption{Flowchart of the Hooke-Jeeves direct search method}
\label{fig:hj}
\centering
\vspace{1em}
\scalebox{.8}{
\includegraphics{images/chapter4_hjmain.tikz}}
\end{figure}

\begin{figure}[H]
\caption{Flowchart of the Exploratory function in the Hooke-Jeeves direct search method}
\label{fig:hje}
\centering
\vspace{1em}
\scalebox{.8}{
\includegraphics{images/chapter4_hjexp.tikz}}
\end{figure}

\begin{algorithm}[H]
\caption{Hooke-Jeeves pattern search method}
\label{alg:hj}
\begin{algorithmic}[1]
\footnotesize     
\State Choose $s^c, \rho, h_{\mathrm{min}}$
\While{$N_{FE} < N_{FE}^{hj}$ \textbf{and} $h > h_{\mathrm{min}}$}  \Comment{Stopping criteria}
\State $s =$ \Call{Exploratory}{$s^c,h$}
\If{$J(s) < J(s^c)$}
\State $s^{\circ} = s + (s - s^c)$
\If{$J(s^{\circ}) < J(s)$}
\State $s^c = s^{\circ}$
\Else
\State $s^c = s$
\EndIf
\Else
\State $h = h \times \rho$
\EndIf
\EndWhile
\State \Return $s^c$
\end{algorithmic}
\end{algorithm}

\begin{algorithm}[H]
\caption{Exploratory function}
\label{alg:hjexploratory}
\begin{algorithmic}[1]
\footnotesize
\Function{Exploratory}{$s^c,h$}
\State $s = s^c$
\For {$d \gets 1$ to $D$}
\If{$J(s + h v_{_d}) < J(s)$}
\State $s = s + h v_{_d}$
\ElsIf{$J(s - h v_{_d}) < J(s)$}
\State $s = s - h v_{_d}$
\EndIf
\EndFor
\State \Return $s$
\EndFunction
\end{algorithmic}
\end{algorithm}

\subsection{Exploratory move}
In the exploratory move, the solution is changed adding and subtracting each time, a column $v_{_d}$ of the search direction matrix $V$, scaled by a step size $h$.

When $V$ is the identity matrix, the modification is performed on the $d$-th element of the solution $s$ each time. This process is done for all the dimensions of the problem. The original solution is compared with both solutions created, and the best among them will be returned.

\subsection{Pattern move}

A new pattern point $s^{\circ}$ is calculated as follows:
%
\begin{equation}
    s^{\circ} = s + \left(s - s^c \right).
\end{equation}
%
where $s$ is the solution obtained from the exploratory move, and $s^c$ is the current solution. If the pattern point $s^{\circ}$ is better than $s^c$, $s^c$ will be replaced with $s^{\circ}$. If there is no improvement, the step size $h$ is reduced in $\rho$ times.

\subsection{Stopping criteria}

A minimum step size $h_{min}$ and a maximum number of function evaluations $N_{FE}^{hj}$ are defined as stopping criteria.