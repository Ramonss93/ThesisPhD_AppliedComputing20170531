\chapter{Hybrid Algorithms for solving the Damage Identification problem}
\label{chp:6}
\epigraph{\textit{Veni, vidi, vici}}{Gaius Julius Caesar}

In the last chapters was presented the forward model that will be used in the inverse problem. 

This chapter will present the hybrid algorithms that will be used for solving the optimization problem. First, will be presented a classification of the optimization algorithms, as well as the concepts of metaheuristic and hybrid algorithms, and the classification of these methods. Then, all the algorithms used in the solution of the problem will be introduced: Multi-Particle Collision Algorithm, $q$-gradient, Opposition-based Optimization, and Hooke-Jeeve direct search method.

\section{Optimization algorithms}

The choice of an appropriate optimization algorithm depends on the optimization problem. 

\autoref{fig:mono} shows a classification for the mono-objective optimization algorithms \cite{Siarry2016}. Following this classification, optimization algorithms can be distinguish in two main branches: \textbf{combinatorial} or \textbf{continuous}. For combinatorial optimization, exist specialized heuristics that are completely dedicated to the problem, and metaheuristics. For continuous optimization, exists a \textbf{linear} approach (with linear programming), distinguished from the \textbf{nonlinear} case. If there is a low number of local minima, a \textbf{local} method should be used, which may or may not use the gradient for searching the objective function. If a high number of local minima exists, a \textbf{global} method should be used, in which are included the traditional methods and the metaheuristics.

\begin{figure}[H]
    \caption{Classification of mono-objective optimization algorithms}
    \label{fig:mono}
    \centering
    \includegraphics{images/chapter4_mono.tikz}
    {\footnotesize SOURCE: Adapted from \cite{Siarry2016}}
\end{figure}

\subsection{Metaheuristic algorithms}

Stochastic optimization has become an important tool to solve multi-modal optimization problems. Sometimes is hard to compute the gradient of the objective function, or even there is no derivative of such function. Most of the stochastic methods do not need the gradient information, or other internal information of the process/system, to be applied. Stochastic methods use random processes to generate new solutions, and facilitate the \textbf{exploration} (global search) in the search space, at the same time that \textbf{exploitation} (local search) is made by some methods. The entire search space can be visited by generating new randomly candidate solutions, while an intense search is done in the neighborhood of this candidate solution -- this searching can be applied to some selected candidates.

With recent advances in the computer science area, many techniques have been developed in the sub-area of stochastic optimization methods. Those algorithms are called to improve the exploration of the search space, making it more efficient, expectedly converging more quickly to the global optimum.

In \citeonline{wolpert1997no}, with their \textit{No Free Lunch} Theorem, established that \say{for any algorithm, any elevated performance over one class of problems is offset by performance over another class}.

Metaheuristic algorithms are powerful tools within the approximated methods that can solve hard optimization problems that could not be solved by deterministic optimization algorithms in a reasonable time \cite{yang2010nature, lin2012review}.

The term metaheuristic was first used by \citeonline{glover1986future}, and comes from the composition of two Greek words: meta and \textit{heuriskein} . A good definition of metaheuristic is given by \citeonline{osman1996metaheuristics}:

\begin{quote}
\say{A metaheuristic is formally defined as an iterative generation process which guides a subordinate heuristic by combining intelligently different concepts for exploring and exploiting the search space, learning strategies are used to structure information in order to find efficiently near-optimal solutions.}
\end{quote}

A huge number of metaheuristics can be found in the literature \cite{du2016search, FisterYFBF13, sorensen2017history}.

The two main features of the metaheuristic algorithms are the \textbf{intensification}, also called exploitation, and the \textbf{diversification}, also called exploration. The exploration phase is responsible for efficiently exploring the search space, while the exploitation phase searches within the current best solutions neighborhood, and selects the best solutions \cite{blum2003metaheuristics,gandomi2013metaheuristic,Yang2014}.

\citeonline{blum2003metaheuristics} summarize some fundamental properties which characterize metaheuristics:

%\begin{quotation}
\begin{itemize}
\item \say{Metaheuristics are strategies that \say{guide} the search process.}
\item \say{The goal is to efficiently explore the search space in order to find (near-) optimal solutions.}
\item \say{Techniques which constitute metaheuristic algorithms range from simple local search procedures to complex learning processes.}
\item \say{Metaheuristic algorithms are approximate and usually non-deterministic.}
\item \say{They may incorporate mechanisms to avoid getting trapped in confined areas of the search space.}
\item \say{The basic concepts of metaheuristics permit an abstract level description.}
\item \say{Metaheuristics are not problem-specific.}
\item \say{Metaheuristics may make use of domain-specific knowledge in the form of heuristics that are controlled by the upper level strategy.}
\item \say{Todays more advanced metaheuristics use search experience (embodied in some form of memory) to guide the search.}
\end{itemize}
%\end{quotation}

Meta-heuristics can be classified in different ways. Among these methods, there are two main categories:

\begin{description}[style=sameline]
    \item[Single-solution] does the search by using only one solution at time; and
    \item[Population-based] uses a set of solutions for exploiting the search space.
\end{description}

Other classification attends their inspiration:

\begin{description}[style=sameline]
    \item[Evolution] Evolutionary Programming (EP) \cite{Fogel:1999:ITS:317034}, Evolution Strategies (ES) \cite{beyer2013theory}, Genetic Algorithms (GA) \cite{Holland:1992:ANA:129194}, Genetic Programming (GP) \cite{langdon:2010:GPEM}, Differential Evolution (DE) \cite{das2011differential}, Cultural Algorithms (CA) \cite{reynolds1994introduction}, and Biogeography-Based Optimization (BBO) \cite{simon2008biogeography};
    \item[Swarm intelligence] Particle Swarm Optimization (PSO) \cite{eberhart1995new,kennedy2010particle}, Ant Colony Optimization (ACO) \cite{dorigo2010ant,dorigo2006ant}, Artificial Bee Optimization (ABC) \cite{karaboga2007powerful}, Bacterial foraging optimization (BFO) \cite{das2009bacterial}, Intelligent Water Drops (IWS) \cite{shah2009intelligent}, and Artificial Immune Systems (AIS) \cite{de2002artificial};
    \item[Human-based] Memetic Algorithms (MA) \cite{Moscato89onevolution}, and Harmony search (HS) \cite{geem2001new}.
    \item[Sciences-based] Simulated Annealing (SA) \cite{kirkpatrick1983optimization}, Particle Collision Algorithm (PCA) \cite{sacco2005new} and Multi-Particle Collision Algorithm (MPCA) \cite{da2008new}.
    \item[Not inspired in nature] Local Search (LS), Greedy Heuristic (GH), Simulated Annealing (SA), Tabu Search (TS), and Iterated Local Search (ILS).
\end{description}

\subsection{Hybrid algorithms}

Hybrid metaheuristics are methods that combine a metaheuristic with other optimization approaches, such as exact methods \cite{jourdan2009hybridizing}, algorithms from mathematical programming, constraint programming, machine learning, or even artificial intelligence \cite{raidl2006unified, raidl2010metaheuristic}.

This cooperation can be in an easy way, where the local method refines the solution obtained from the metaheuristic. Also, a more complex way of hybrid algorithms can be found, in which the single methods are intermingled.

Hybridizing different algorithmic concepts allows obtaining a better performance, exploiting and combining the advantages of single strategies \cite{Blum2010}.

Hybrid algorithms can be classified into two main types \cite{talbi2002taxonomy, jourdan2009hybridizing}: \textbf{low-level}, with a functional composition, where a given function of a metaheuristic is replaced by another method or \textbf{high-level}: there are no composition of the  different algorithms, retaining their own identities; and working as a \textbf{relay}, where one algorithm takes as its inputs the output of the previous algorithms, working in series, like a pipeline, or a \textbf{teamwork}, using cooperative optimization models.

Another classification was presented by \citeonline{talbi2002taxonomy}, where divided the hybrid algorithm in \textbf{homogeneous}, when all the combined algorithms use the same metaheuristic, or \textbf{heterogeneous}, in which different metaheuristics are used; and \textbf{global}, with all the algorithms searching in the whole search space, or \textbf{partial}, when the problem is divided into sub-spaces, and the algorithms perform the search in its search space; and \textbf{general}, when all the algorithms solve the same optimization problem, or \textbf{specialist}, when each algorithm solve a different problem.

A new classification was made by \citeonline{ting2015}, separating the hybrid algorithms into two groups according to their taxonomy:
\vspace{1em}
%
\begin{itemize}
    \item Collaborative hybrids combine two or more algorithms that could work in three ways:
    \vspace{1em}
    \begin{itemize}
        \item Multi-stage, combining two stages: a global search followed by a local search;
        \item Sequential, running both algorithms alternatively until a stopping criterion is met; or,
        \item Parallel, where the algorithms run simultaneously over the same population.
    \end{itemize}
    \item Integrative hybrids, where a master algorithm has other algorithm embedded working in two possible ways:
    \vspace{1em}
    %
    \begin{itemize}
        \item with a full manipulation of the population at every iteration, or
        \item with the manipulation of a portion of the population.
    \end{itemize}
\end{itemize}

\section{Multi-Particle Collision Algorithm}
\label{sec:mpca}

Multi-Particle Collision Algorithm (MPCA) is a metaheuristic inspired by the physics of nuclear particle collision reactions created by \citeonline{Luz2008} based on the Particle Collision Algorithm (PCA) from \citeonline{sacco2005new}. It is based on the scattering and absorption phenomena that occur inside the nuclear reactor. Scattering is when an incident particle is scattered by a target nucleus, while absorption is when an incident particle is absorbed by the target nucleus, as depicted in \autoref{fig:phenomena}.

\begin{figure}[H]%
\caption{Phenomena inside a nuclear reactor that inspire MPCA: \subref{fig:scattering} scattering; and \subref{fig:absorption} absorption.}%
\label{fig:phenomena}%
\vspace{1em}
\centering
\subfigure[][]{%
\label{fig:scattering}%
\begin{minipage}{0.45\textwidth}
\centering
\scalebox{0.6}{
\includegraphics{images/chapter4_scattering.tikz}}%
\end{minipage}}
\hspace{8pt}%
\subfigure[][]{%
\label{fig:absorption}%
\begin{minipage}{0.45\textwidth}
\centering
\scalebox{0.6}{
\includegraphics{images/chapter4_absorption.tikz}}
\end{minipage}}%
\end{figure}

In MPCA, a set of particles (i.e. candidate solutions), travels through the search space. There are three primary functions: perturbation, exploitation, and scattering. New solutions are created perturbing the particles. After the perturbation is applied, if the new position of the particle is better than the previous one, an intensification is made in its neighborhood, looking for improving, even more, the solution found. If the new particle position is worse than the previous one, two possible processes can be done, depending on predefined probability: a new random position is generated, or a local search is performed in the neighborhood of the particle. The flowchart of the MPCA is shown in \autoref{fig:mpca}, and the pseudo-code is presented in \autoref{alg:mpca}.

\begin{figure}[H]
\caption{Flowchart of the Multi-Particle Collision Algorithm}
\label{fig:mpca}
\centering
\vspace{1em}
\scalebox{0.8}{
\includegraphics{images/chapter4_mpca.tikz}
}
\end{figure}

\begin{algorithm}[H]
\caption{Multi-Particle Collision Algorithm}
\label{alg:mpca}
\footnotesize
\begin{algorithmic}[1]
\State Set MPCA control parameters ($N_{processors}$, $N_{particles}$, $N_{FE}^{mpca}$, $N_{FE}^{blackboard}$, $s^l$, $s^u$, $R^{inf}$, $R^{sup}$)
\For {$i \gets 1 \; \textbf{to} \; N_{processors}$} \Comment{Initial set of particles}
\State $N_{FE{_i}} = 0$, $N_{FE_i}^{lastUpdate} = 0$
\For {$j \gets 1 \; \textbf{to} \; N_{particles}$}
\State $s_{i,j} = $ \Call{RandomSolution}{}
\State $N_{FE{_i}} = N_{FE{_i}} + 1$
\EndFor
\State $s^b_i$ = \Call{UpdateBlackboard}{} \Comment{Initial blackboard}
\EndFor
\While{$N_{FE{_{total}}} < N_{FE}^{mpca}$} \Comment{Stopping criteria}
\State $N_{FE{_{total}}} = 0$
\For {$i \gets 1 \; \textbf{to} \; N_{processors}$}
\For {$j \gets 1 \; \textbf{to} \; N_{particles}$}
\State $s^\star_{i,j}$ = \Call{Perturbation}{$s_{i,j}$}
\If {$J(s^\star_{i,j}) < J(s_{i,j})$}
\State $s_{i,j}$ = $s^\star_{i,j}$
\State $s_{i,j}$ = \Call{Exploration}{$s_{i,j}$}
\Else
\State $s_{i,j}$ = \Call{Scattering}{$s_{i,j}$, $s^\star_{i,j}$, $s^b_i$}
\EndIf
\If {$J(s_{i,j})< J(s^b_i)$}
\State $s^b_i$ = $s_{i,j}$
\EndIf
\EndFor
\If {$N_{FE{_i}}$ - $N_{FE_i}^{lastUpdate} > N_{FE}^{blackboard}$}
\For {$i \gets 1 \; \textbf{to} \; N_{processors}$} \Comment{Blackboard}
\State $s^b_i$ = \Call{UpdateBlackboard}{}
\State $N_{FE_i}^{lastUpdate} = N_{FE{_i}}$
\EndFor
\EndIf
\State $N_{FE{_{total}}} = N_{FE{_{total}}} + N_{FE{_i}}$
\EndFor
\EndWhile
\For {$i \gets 1 \; \textbf{to} \; N_{processors}$} \Comment{Final blackboard}
\State $s^b_i$ = \Call{UpdateBlackboard}{} 
\EndFor
\State \Return $s^b_1$
\end{algorithmic}
\end{algorithm}

MPCA has been used in the solution of some optimization problems such as fault diagnosis \cite{echevarria2012aplicacion}, automatic configuration of neural networks applied to atmospheric temperature profile identification \cite{sambattiautomatic}, data assimilation \cite{anochiself} and climate prediction \cite{anochi2014optimization}, as well as in the solution of an inverse radiative problem \cite{hdez2015}, obtaining good results.

A parallel version of MPCA using OpenMPI\footnote{\url{https://www.open-mpi.org/}} was implemented in FORTRAN 90, in a multiprocessor architecture with distributed memory machine.

\subsection{Initial set of particles}

The initial set of particles is constituted by $N_{particles}$ particles. The generation of candidate solutions in the initial set of particles can be performed in two ways: using good solutions known so far, or creating the solution randomly within the search space.

For generating a random solution, each coordinate $d$ of a solution $s =  \left( s_{_1}, \cdots, s_{_n} \right) \in \R^n$ is found as:
%
\begin{equation}
    \label{eq:newrandom}
    s_{_d} = s^l_{_d} + (s^u_{_d} - s^l_{_d}) \times rand(0,1)
\end{equation}

\subsection{Perturbation function}

The \texttt{Perturbation} function performs a random variation of a particle within a defined range. A perturbed particle $s^\star$ is completely defined by its coordinates:
%
\begin{equation}
s_{_d}^\star = s_{_d} + \left(s^u_{_d} - s_{_d}\right) R - \left(s_{_d} - s^l_{_d}\right) \left(1 - R\right),
\end{equation}
%
where $s$ is the particle to be perturbed, $s^u_{_d}$ and $s^l_{_d}$ are the upper and the lower limits of the defined search space, respectively, and $R = rand(0,1)$.

The pseudo-code of this function is shown in \autoref{alg:mpcaPerturbation}.

\begin{algorithm}[H]
\caption{Perturbation function}
\label{alg:mpcaPerturbation}
\footnotesize
\begin{algorithmic}[1]
\Function{Perturbation}{$s$}
\For {$d \gets 1 \; \textbf{to} \; D$}
\State $R = rand(0,1)$
\State $s_{_d}^\star = s_{_d} + \left(s^u_{_d} - s_{_d}\right) R - \left(s_{_d} - s^l_{_d}\right) \left(1 - R\right)$
\If {{$s_{_d}^\star > s^u_{_d}$}}
\State $s_{_d}^\star = s^u_{_d}$
\ElsIf {{$s_{_d}^\star < s^l_{_d}$}}
\State $s_{_d}^\star = s^l_{_d}$
\EndIf
\EndFor
\State $N_{FE} = N_{FE} + 1$
\State \Return $s^\star$
\EndFunction
\end{algorithmic}
\end{algorithm}

If $J(s^\star) < J(s)$, then the particle $s$ is replaced with $s^\star$, and the \textsc{Exploitation} function (see \autoref{sec:exploitation}) is activated, performing an exploitation in the neighborhood of the particle. If the new perturbed particle $s^\star$ is worse than the current particle $s$, the \textsc{Scattering} function (see \autoref{sec:scattering}) is launched.

\subsection{Exploitation function}
\label{sec:exploitation}

In this stage, the algorithm performs a series of $N_{FE}^{exploitation}$ small perturbations, computing a new particle $s^\S$ each time, using the following equation for each coordinate:
%
\begin{equation}
s_{_d}^\S = s_{_d} + \left(u - s_{_d}\right) R- \left(s_{_d} - l\right)\left(1 - R\right),
\end{equation}
%
where $u = s_{_d} \times rand(1,R^{sup})$ is the small upper limit, and $l = s_{_d} \times rand(R^{inf},1)$ is the small lower limit, with a superior value $R^{sup}$ and a inferior value $R^{inf}$ for the generation of the random number.

The pseudo-code of this function is shown in \autoref{alg:mpcaExploitation} and \autoref{alg:mpcaSmallPerturbation}.

\begin{algorithm}[H]
\caption{Exploitation function}
\label{alg:mpcaExploitation}
\footnotesize
\begin{algorithmic}[1]
\Function{Exploitation}{$s$}
\For {$n \gets 1 \; \textbf{to} \; N_{FE}^{exploitation}$}
\State $s^\star$ = \Call{SmallPerturbation}{$s$}
\State $N_{FE} = N_{FE} + 1$
\If {$J(s^\star) < J(s)$}
\State $s = s^\star$
\EndIf
\EndFor
\State \Return $s$
\EndFunction
\end{algorithmic}
\end{algorithm}

\begin{algorithm}[H]
\caption{Small Perturbation function}
\label{alg:mpcaSmallPerturbation}
\footnotesize
\begin{algorithmic}[1]
\Function{SmallPerturbation}{$s$}
\For {$d \gets 1 \; \textbf{to} \; D$}
\State $u = s_{_d} \cdot rand(1,R^{sup})$
\State $l = s_{_d} \cdot rand(R^{inf},1)$
\State $R = rand(0,1)$
\If {{$u > s^u_{_d}$}}
\State $u = s^u_{_d}$
\EndIf
\If {{$l < s^l_{_d}$}}
\State $l = s^l_{_d}$
\EndIf
\State $s_{_d}^\S = s_{_d} + \left(u - s_{_d}\right) R- \left(s_{_d} - l\right)\left(1 - R\right)$
\EndFor
\State \Return $s^\S$
\EndFunction
\end{algorithmic}
\end{algorithm}

\subsubsection{Exploitation in a single random dimension each time}

The exploitation function is modified, performing a small perturbation in only one dimension chosen randomly, as shown in \autoref{alg:mpcaSmallPerturbationModified}.

\begin{algorithm}[H]
\caption{Small Perturbation in a single random dimension function}
\label{alg:mpcaSmallPerturbationModified}
\footnotesize
\begin{algorithmic}[1]
\Function{SmallPerturbation}{$s$}
\State $d = rand(0,1) \times (D - 1) + 1;$
\State $u = s_{_d} \cdot rand(1,R^{sup})$
\State $l = s_{_d} \cdot rand(R^{inf},1)$
\State $R = rand(0,1)$
\If {{$u > s^u_{_d}$}}
\State $u = s^u_{_d}$
\EndIf
\If {{$l < s^l_{_d}$}}
\State $l = s^l_{_d}$
\EndIf
\State $s_{_d}^\S = s_{_d} + \left(u - s_{_d}\right) R- \left(s_{_d} - l\right)\left(1 - R\right)$
\State \Return $s^\S$
\EndFunction
\end{algorithmic}
\end{algorithm}

\subsection{Scattering function}
\label{sec:scattering}

The \textsc{Scattering} function works as a Metropolis scheme. The pseudo-code is presented in \autoref{alg:mpcaScattering}.

\begin{algorithm}[H]
\caption{Scattering function}
\label{alg:mpcaScattering}
\footnotesize
\begin{algorithmic}[1]
\Function{Scattering}{$s$, $s^\star$, $s^b$}
\State Find $p^s$
\If {{$p^s > rand(0,1)$}}
\State $s^n$ = \Call{RandomSolution}{}
\State $N_{FE} = N_{FE} + 1$
\Else
\State $s$ = \Call{Exploration}{$s$}
\EndIf
\State \Return $s$
\EndFunction
\end{algorithmic}
\end{algorithm}

The particle $s$ is replaced by a new random solution $s^n$ using \autoref{eq:newrandom}, or the exploitation is made, with a given probability.

The scattering probability can be found in some different ways, such as:

\begin{itemize}
    \item Truncated exponential distribution:
    \begin{equation}
    p^s =1-\dfrac{J(s^b)}{J(s)}
    \end{equation}
    \item Cauchy distribution:
    \begin{equation}
    p^s = \frac{1}{\pi \gamma \left(1 + \left(\dfrac{J(s) - J(s^b)}{\gamma}\right)^2\right)},\quad \gamma = 1
    \end{equation}
\end{itemize}

\subsection{Blackboard updating function}

A mechanism called \textit{blackboard updating} allows sending the best solution overall reached in certain moments of the algorithm to the other particles. That moment is determined by the number of function evaluations. Each $N_{FE}^{blackboard}$ evaluations, an update of the best particle will be done.

Each $N_{FE}^{blackboard}$ function evaluations, the best particle overall $s^{\Diamond}$ is elected and sent to all the particles, updating the reference $s^b$.

\begin{algorithm}[H]
\caption{UpdateBlackboard function using MPI}
\label{alg:mpcaBlackboard}
\footnotesize
\begin{algorithmic}[1]
\Function{UpdateBlackboard}{$s^b$}
\If {is master processor}
\State $s^{\Diamond} = s^b$
\For {$i \gets 1 \; \textbf{to} \; N_{processors}$}
\State $s$ = \Call{MPI\_RECEIVE}{$i$} \Comment{\parbox[t]{.55\linewidth}{Receive best particle from each processor}}
\If {{$J(s) < J(s^{\Diamond})$}}
\State $s^{\Diamond} = P$ \Comment{\parbox[t]{.55\linewidth}{Update the best particle overall}}
\EndIf
\EndFor
\State \Call{MPI\_BROADCAST}{$s^{\Diamond}$} \Comment{\parbox[t]{.55\linewidth}{Send best particle overall to the other processors}}
\Else \Comment{\parbox[t]{.55\linewidth}{Other processors}}
\State \Call{MPI\_SEND}{$s^b$} \Comment{\parbox[t]{.55\linewidth}{Send the self best particle to the master processor}}
\State \Call{MPI\_BROADCAST}{$s^{\Diamond}$} \Comment\parbox[t]{.55\linewidth}{{Wait for the best particle overall}}
\EndIf

\State \Return $s^{\Diamond}$
\EndFunction
\end{algorithmic}
\end{algorithm}

\subsection{Stopping criteria}

As for stopping criterion, a maximum number of function evaluations ($N_{FE}^{mpca}$) is defined. 

\section{Opposition-Based Optimization and some mechanisms derived}
\label{sec:obl}

The Opposition-based Learning (OBL) mechanism was created by \citeonline{Tizhoosh2005} in 2005. The idea of OBL is to consider the opposite of a candidate solution, which has a certain probability of being closer to the global optimum.

Some mechanisms derived from OBL have been developed, such as \textbf{Quasi-opposition} (QO), \textbf{Quasi-reflection} (QR), \textbf{Center-based} sampling (CB), and Rotation-based Learning (RBL) \cite{ergezer2009oppositional,rahnamayan2007quasi,tizhoosh2008oppositional}. QO reflects a point to a random point between the center of the domain and the opposite point. QR projects the point to a random point between the center of the domain and itself. CB creates a point between itself and its opposite.

In a short amount of time, these mechanisms have been utilized in different soft computing areas, improving the performance of various techniques of Computational Intelligence, such as metaheuristics, artificial neural networks, fuzzy logic, and other applications \cite{xu2014review}.

For better understanding the mechanism, it is necessary to define the concept of some specific numbers.

\begin{definition}
Let $s \in \left[s^l, s^u\right]$ be a real number, and $c = (s^l + s^u) / 2$. The opposite number $s_o$, the quasi-opposite number $s_{qo}$, the quasi-reflected number $s_{qr}$, and the center-based sampled number $s_{cb}$ are defined as:
%
\begin{align}
s_o =& s^l + s^u - s;\\
s_{qo} =& rand(s_o,c);\\
s_{qr} =& rand(c,s);\\
s_{cb} =& rand(s_o,s).
\end{align}

\end{definition}

\autoref{fig:number} shows a graphical representation of these numbers.

\begin{figure}[H]
\caption{Graphical representation of the opposition number ($s_o$), quasi-opposite number ($s_{qo}$), quasi-reflected number ($s_{qr}$) and center-based sampled number ($s_{cb}$) from the original number $s$}
\label{fig:number}
\centering
\vspace{1em}
\includegraphics{images/chapter4_opposition.tikz}
\end{figure}

\begin{definition}
\label{def:obl}
Let $s=(s_{_1}, \cdots, s_{_D}) \in \R^D$ be a point, $s_{_d} \in \left[s^l_{_d}, s^u_{_d}\right], \forall d \in \left(1, \dots, n\right)$. The opposite point $s_o$, the quasi-opposite point $s_{qo}$, the quasi-reflected point $s_{qr}$, and the center-based sampled point $s_{cb}$ are completely defined by their coordinates:
%
\begin{align}
s_{o_d} =& s^l_{_d} + s^u_{_d} - s_{_d};\\
s_{qo_d} =& rand(s_{o_d},c_{_d});\\
s_{qr_d} =& rand(c_{_d},s_{_d});\\
s_{cb_d} =& rand(s_{o_d},s_{_d}).
\end{align}

\end{definition}

The Rotation-Based Learning (RB) mechanism is another extension of the OBL \cite{Liu2014}, and the Rotation-Based Sampling (RBS) is a combination of the Center-Based Sampling and RBL mechanisms.

\begin{definition}
Let $s \in \left[s^l, s^u\right]$ be a real number, and $c = (s^l + s^u) / 2$ be the center. Draw a circle with center $c$ and radius $c-s^l$. The point $(s, 0)$ is projected on the circle. Defining the quantity from the original number to the center $u = s - c$, the lenght from the original number to the corresponding intersection point $l$ on the circle $v = \sqrt{ \left(s - s^l \right)\left( s^u - s \right)}$, and the deflection angle $\beta = \beta_0 \, \mathcal{N}(1, \delta)$, with mean $\beta_0$ and standard deviation $\delta$. The rotation number $s_r$ is defined as:
%
\begin{equation}
s_{r} = c + u \times \cos \beta - v \times \sin \beta
\end{equation}

\end{definition}

\begin{definition}
The rotation-based sampling number $s_{rbs}$ is defined as:
\begin{equation}
s_{rbs} = rand(s_r,s)
\end{equation}

\end{definition}

The geometric representation in a 2D-space of the Rotation and the Rotation-Based Sampling numbers is shown in Figure~\ref{fig:rbl}.

\begin{figure}[H]
\caption{Geometric interpretation of the Rotation ($s_r$) and the Rotation-based Sampling ($s_{rbs}$) numbers in 2D space}
\label{fig:rbl}
\centering
\vspace{1em}
\begin{tikzpicture}

\draw[-latex,thick] (-2.5,0)--(2.5,0) node[right]{$x$};
\draw[-latex,thick] (-2.2,-2.0)--(-2.2,2.0) node[above]{$y$};

\coordinate (O) at (0,0);

\draw[dotted] (O) node[
label={[xshift=-2.0cm, yshift=-0.7cm]\footnotesize $s^l$},
label={[xshift=2.0cm, yshift=-0.7cm]\footnotesize $s^u$},
label={[xshift=0.0cm, yshift=-0.7cm]\footnotesize $c$},
label={[xshift=1.7cm, yshift=-0.7cm]\footnotesize $s$},
label={[xshift=1.9cm, yshift=0.7cm]\footnotesize $l$},
label={[xshift=-1.7cm, yshift=-0.75cm]\footnotesize $s_r$}] {} circle [radius=\myrad,dotted];

\draw[-,semithick] (0,-0.1) -- (0,0.1);
\draw[-,semithick] (-2,-0.1) -- (-2,0.1);
\draw[-,semithick] (2,-0.1) -- (2,0.1);

\draw[-,semithick] (\myang:\myrad)+(0,-1.1cm) -- (\myang:\myrad);
\draw[-,semithick] (\myang+130:\myrad)+(0,-0.8cm) -- (\myang+130:\myrad);

\coordinate (a) at (-1.85,-0.5);
\coordinate (d) at (1.7,-0.5);
\draw[decorate,decoration={brace,amplitude=5pt,raise=0.5pt},color=black,rotate=90] (d) -- (a) node [midway,yshift=-11pt] {\footnotesize $s_{rbs}$};

\draw 
  (\myrad,0) coordinate (xcoord) -- 
  node[midway,below] {} (O) -- 
  (\myang:\myrad) coordinate (slcoord)
  pic [draw,-latex,angle radius=1cm] {angle = xcoord--O--slcoord};

\draw 
  (\myrad,0) coordinate (xcoord) -- 
  node[midway,below] {} (O) -- 
  (\myang+130:\myrad) coordinate (slcoord)
  pic [draw,-latex,angle radius=0.8cm] {angle = xcoord--O--slcoord};

\node at ($(O)+(15:12mm)$) {\footnotesize $\theta$};
\node at ($(O)+(85:10mm)$) {\footnotesize $\theta + \beta$};

\end{tikzpicture}
\end{figure}

Similarly to \autoref{def:obl}, the concepts of rotation and rotation-based sampling points are enunciated:

\begin{definition}
Let $s = (s_{_1}, \cdots, s_{_D}) \in \R^D$ be a point, and $s_{_d} \in \left[s^l_{_d}, s^u_{_d}\right], \forall d \in \left(1, \cdots, D \right)$. The rotation point $s_r$, is completely defined by their coordinates:
%
\begin{equation}
s_{r_{d}} = c_{_d} + u_{_d} \times \cos \beta - v_{_d} \times \sin \beta
\end{equation}

\end{definition}

\begin{definition}
Let $s = (s_{_1}, \cdots, s_{_D}) \in \R^D$ be a point, and $s_{_d} \in \left[s^l_{_d}, s^u_{_d}\right], \forall d \in \left(1, \cdots, D \right)$. The rotation-based sampling number $s_{rbs}$ is completely defined by their coordinates:
%
\begin{equation}
s_{rbs_d} = rand(s_{r_d},s_{_d})
\end{equation}

\end{definition}

\begin{definition} \textbf{Opposition-based Optimization --}
Let $s \in \R^D$ be a point (\textit{i.e.}, candidate solution), and $s_o$ a opposite point of $s$ (i.e., opposite candidate solution). If $J(s_o) \leq J(s)$, then the point $s$ can be replaced with $s_o$, which is better, otherwise it will maintain its current value.
\end{definition}

The solution and the opposite solution are evaluated simultaneously, and the optimization process will continue with the better one.

The same idea of the Opposition-based Optimization is applicable for the other mechanisms. To abbreviate, when the text refers to all these mechanisms, will be used without distinction OBO or Opposition. \autoref{alg:3} shows a pseudo-code of a possible computational implementation of the Opposition.

\begin{algorithm}[H]
\centering
\caption{Opposition function}
\label{alg:3}
\footnotesize
\begin{algorithmic}[1]
\Function{Opposition}{$s$}
\For {$d \gets 1 \; \textbf{to} \; D$}
\State Obtain $ s^l_{_d} $, $ s^u_{_d} $
\State $ c_{_d} = \dfrac{\left( s^l_{_d} + s^u_{_d} \right)}{2} $ \Comment{Calculate center of the search space}
\State $ s_{o_d}  = s^l_{_d} + s^u_{_d} - s_{_d} $
\Comment{Calculate opposite point}
\State $ R = rand(0,1) $
\Switch{type}
\Case{quasi-opposition}
\If{$ s_{_d} < c_{_d} $}
\State $ s_{o_d} = c_{_d} + R \left(s_{o_d} - c_{_d} \right) $
\Else
\State $ s_{o_d} = s_{o_d} + R \left(c_{_d} - s_{o_d} \right) $
\EndIf
\EndCase
\Case{quasi-reflected}
\If{$ s_{_d} < c_{_d} $}
\State $ s_{o_d} = s_{_d} + R \left(c_{_d} - s_{_d} \right) $
\Else
\State $ s_{o_d} = c_{_d} + R \left(s_{_d} - c_{_d} \right) $
\EndIf
\EndCase
\Case{center-based sampling}
\If{$ s_{_d} < c_{_d} $}
\State $ s_{o_d} = s_{_d} + R \left(s_{o_d} - s_{_d} \right) $
\Else
\State $ s_{o_d} = s_{o_d} + R \left(s_{_d} - s_{o_d} \right) $
\EndIf
\EndCase
\Case{rotation-based sampling}
\State $ u_{_d} = s_{_d} - c_{_d} $
\State $ v_{_d} = \sqrt{\left(s_{_d} - s^l_{_d} \right)\left( s^u_{_d} - s_{_d} \right)} $
\State $ s_{r_d} = c_{_d} + u_{_d} \times \cos \beta - v_{_d} \times \sin \beta$

\If{$ s_{_d} < c_{_d} $}
\State $ s_{o_d} = s_{_d} + R \left(s_{r_d} - s_{_d} \right) $
\Else
\State $ s_{o_d} = s_{r_d} + R \left(s_{_d} - s_{r_d} \right) $
\EndIf
\EndCase
\Default \Comment{Opposition}
\State $ s_{o_d} =  s_{o_d} $
\EndDefault
\EndSwitch
\EndFor
\State $N_{FE} = N_{FE} + 1$
\If {$J(s_o) < J(s)$}
\State \Return $s_o$
\Else
\State \Return $s$
\EndIf
\EndFunction
\end{algorithmic}
\end{algorithm}

\section{Multi-Particle Collision Algorithm with Opposition-Based Optimization derived mechanisms}
\setcounter{equation}{0}
\label{sec:4}

The hybrid of MPCA with Opposition is defined as follows. The pseudo-code of the computational implementation of the generic algorithm is shown in \autoref{alg:ompca}.

\begin{algorithm}[H]
\caption{Multi-Particle Collision Algorithm with Opposition}
\label{alg:ompca}
\footnotesize
\begin{algorithmic}[1]
\State Set MPCA control parameters ($N_{processors}$, $N_{particles}$, $N_{FE}^{mpca}$, $N_{FE}^{blackboard}$, $s^l$, $s^u$, $R^{inf}$, $R^{sup}$)
\For {$i \gets 1 \; \textbf{to} \; N_{processors}$} \Comment{Initial set of particles}
\State $N_{FE_i} = 0$, $N_{FE_i}^{lastUpdate} = 0$
\For {$j \gets 1 \; \textbf{to} \; N_{particles}$}
\State $s^\star_{i,j} = $ \Call{RandomSolution}{}
\State $N_{FE_i} = N_{FE_i} + 1$
\State $s^\star_{i,j} = $ \Call{Opposition}{$s^\star_{i,j}$} \label{alg:line:ompca6}
\EndFor
\EndFor
\For {$i \gets 1 \; \textbf{to} \; N_{processors}$} \Comment{Initial blackboard}
\State $s^b_i$ = \Call{UpdateBlackboard}{} 
\EndFor
\While{$N_{FE_i} < N_{FE}^{mpca}$} \Comment{Stopping criteria}
\For {$i \gets 1 \; \textbf{to} \; N_{processors}$}
\For {$j \gets 1 \; \textbf{to} \; N_{particles}$}
\State $s_{i,j}$ = \Call{Perturbation}{$s^\star_{i,j}$}
\If {$J(s_{i,j}) < J(s^\star_{i,j})$}
\State $s^\star_{i,j}$ = $s_{i,j}$
\State $s^\star_{i,j}$ = \Call{Exploration}{$s^\star_{i,j}$}
\Else
\State $s^\star_{i,j}$ = \Call{Scattering}{$s^\star_{i,j}$, $s_{i,j}$, $s^b_i$}
\EndIf
\If {$J(s^\star_{i,j})< J(s^b_i)$}
\State $s^b_i$ = $s^\star_{i,j}$
\EndIf
\State $s^b_i = $ \Call{Opposition}{$s^b_i$} \label{alg:line:ompca20}
\EndFor
\EndFor
\If {$N_{FE_i}$ - $N_{FE_i}^{lastUpdate} > N_{FE}^{blackboard}$}
\For {$i \gets 1 \; \textbf{to} \; N_{processors}$} \Comment{Final blackboard}
\State $s^b_i$ = \Call{UpdateBlackboard}{}
\State $N_{FE_i}^{lastUpdate} = N_{FE_i}$
\EndFor
\EndIf
\EndWhile
\For {$i \gets 1 \; \textbf{to} \; N_{processors}$} \Comment{Final blackboard}
\State $s^b_i$ = \Call{UpdateBlackboard}{} 
\EndFor
\State \Return $s^b_1$
\end{algorithmic}
\end{algorithm}

\subsection{Initialization using Opposition}

Random number generation is commonly the most used choice to create an initial population. The use of opposition working together with randomness permits to obtain better-starting candidates even when there is no \textit{a priori} knowledge about the solution.

In the proposed hybrid versions, the first step is to create the initial solution for each particle as usual. Next, the opposite solution is calculated within the search space $\left[ s^l_{_d}, s^u_{_d} \right]$. The original solution is substituted by the opposite if the latter has a better fitness (see \autoref{alg:ompca}, line~\ref{alg:line:ompca6}).

\subsection{Traveling in the search space using Opposition}

The application of Opposition on the traveling of the particles in the search space is dependent on the MPCA function being called.

When the \texttt{Perturbation} function is applied on a particle, the opposite particle is calculated at the same time. The best particle among them will be maintained as the new particle (see \autoref{alg:mpcaPerturbationOpposition}, line~\ref{alg:line:mpcaPerturbationOpposition10}). The bounds to create the opposite particle is dynamically reduced to $\left[ s^l_{_d}, s^u_{_d} \right]$, where $s^l_{_d}$ and $s^u_{_d}$ are the minimum and maximum values for each dimension in all the population of particles.

\begin{algorithm}[H]
\caption{Perturbation function with Opposition}
\label{alg:mpcaPerturbationOpposition}
\footnotesize
\begin{algorithmic}[1]
\Function{Perturbation}{$s$}
\For {$d \gets 1 \; \textbf{to} \; D$}
\State $R = rand(0,1)$
\State $s_{_d}^\star = s_{_d} + \left(s^u_{_d} - s_{_d}\right) R - \left(s_{_d} - s^l_{_d}\right) \left(1 - R\right)$
\If {{$s_{_d}^\star > s^u_{_d}$}}
\State $s_{_d}^\star = s^u_{_d}$
\ElsIf {{$s_{_d}^\star < s^l_{_d}$}}
\State $s_{_d}^\star = s^l_{_d}$
\EndIf
\EndFor
\State $N_{FE} = N_{FE} + 1$
\State $s^\star$ = \Call{Opposition}{$s^\star$} \label{alg:line:mpcaPerturbationOpposition10}
\State \Return $s^\star$
\EndFunction
\end{algorithmic}
\end{algorithm}

When the \texttt{Exploration} is performed, an opposite particle is also calculated, with a jumping rate $J_r$ (see \autoref{alg:mpcaExploitationOpposition}, lines~\ref{alg:line:mpcaExploitationOpposition6}). The bounds to create the opposite particle is dynamically reduced to $\left[ s^l_{_d}, s^u_{_d} \right]$, as was done in the \texttt{Perturbation} function.

\begin{algorithm}[H]
\caption{Exploitation function with Opposition}
\label{alg:mpcaExploitationOpposition}
\footnotesize
\begin{algorithmic}[1]
\Function{Exploitation}{$s$}
\For {$n \gets 1 \; \textbf{to} \; N_{FE_{exploitation}}$}
\State $s^\star$ = \Call{SmallPerturbation}{$s$}
\State $N_{FE} = N_{FE} + 1$
\If {$rand(0,1) < J_r$}
\State $s^\star$ = \Call{Opposition}{$s^\star$} \label{alg:line:mpcaExploitationOpposition6}
\EndIf
\If {$J(s^\star) < J(s)$}
\State $s = s^\star$
\EndIf
\EndFor
\State \Return $s$
\EndFunction
\end{algorithmic}
\end{algorithm}

In the \texttt{Scattering} function, if a random particle is created, then the opposite particle is also created using the original bounds $\left[ s^l_{_d}, s^u_{_d} \right]$. The best particle among then will be maintained (see \autoref{alg:mpcaScatteringOpposition}, line~\ref{alg:mpcaScatteringOpposition6}).

\begin{algorithm}[H]
\caption{Scattering function with Opposition}
\label{alg:mpcaScatteringOpposition}
\footnotesize
\begin{algorithmic}[1]
\Function{Scattering}{$s$, $s^\star$, $s^b$}
\State Find $p^s$
\If {{$p^s > rand(0,1)$}}
\State $s^n$ = \Call{RandomSolution}{}
\State $N_{FE} = N_{FE} + 1$
\State $s$ = \Call{Opposition}{$s^n$} \label{alg:mpcaScatteringOpposition6}
\Else
\State $s$ = \Call{Exploration}{$s$}
\EndIf
\State \Return $s$
\EndFunction
\end{algorithmic}
\end{algorithm}

After applying \texttt{Perturbation}, \texttt{Exploration}, and \texttt{Scattering} functions to generate the new solution, the opposite of the best particle heretofore is calculated using the computed limits $\left[ s^l_{_d}, s^u_{_d} \right]$ (see \autoref{alg:ompca}, line~\ref{alg:line:ompca20}).

\section{\texorpdfstring{$q$}{q}-Gradient Method}
\label{sec:qg}

The first concepts of the $q$-calculus were developed by \citeonline{jackson08,jackson10a,jackson10b}. At the beginning of the 20th century appeared the $q$-analogs of functions, series, special numbers, and the $q$-derivative concepts, including the $q$-gradient vector.

The $q$-gradient ($q$G) method can be described as a $q$-analog of the \textit{steepest descent} method that reduces to its classical version whenever the parameter $q=1$. For $q \neq 1$, the search direction is likely to be either descent or not descent, and the method performs a global search.

In the $q$G method, the search process gradually shifts from global in the beginning to almost local search in the end \cite{soterroni12a}.

\begin{definition}
Let $J$ be a differentiable (objective) function of $n$ variables, the $n$ first-order partial $q$-derivatives of $J$ with respect to the variable $s_d$ are given by \autoref{eq:dq} \cite{soterroni2011,soterroni12a,soterroni13}.
%
\begin{equation}
\label{eq:dq}
D_{q_{_d},s_{_d}} J(s) =
\begin{dcases}
\dfrac{J(s_{_1}, \cdots,q_{_d} s_{_d}, \cdots, s_{_n}) - J(s_{_1}, \cdots, s_{_n}) }{q_{_d} s_{_d} -  s_{_d}},  & \text{if} \, s_{_d}  \neq 0 \; \textrm{and} \; q_{_d} \neq 1, \\
\dfrac{\partial J(s)}{\partial s_{_d}}, & \text{otherwise},
\end{dcases}
\end{equation}
%
where $ q =  \left( q_{_1}, \cdots, q_{_n} \right) \in \R^n $.
\end{definition}

When $s_{_d} = 0$ or $q_{_d} = 1$, $\forall d$, the first-order partial $q$-derivative is the classical first-order partial derivative.

\begin{definition}
The $q$-gradient is the vector of the $n$ first-order partial $q$-derivatives of $f$ \cite{soterroni2011,soterroni12a,soterroni13}
%
\begin{equation}
\label{eq:qgradient}
\nabla_{q}J(s) = \left[D_{q_{_1},s_{_1}} J(s) \cdots D_{q_{_d},s_{_d}} J(s) \cdots D_{q_{_n},s_{_n}} J(s)\right]. 
\end{equation}
\end{definition}

The $q$G method uses an iterative procedure that, starting from an initial point $s$, generates a sequence $s$ given by:
%
\begin{equation}
s =  s + \alpha v,
\end{equation}
%
where $v$ is the search direction, and $\alpha$ is the step length or the distance moved along $v$. Thus, the search direction $v$ in the $q$G method is the negative of the $q$-gradient of the objective function $- \nabla_q J(s)$ (\autoref{eq:qgradient}).

The parameters $q_{_d}$ are generated drawing their values from a Gaussian probability distribution, such that $q_{_d} s_{_d}$ has a standard deviation $\sigma$ that decreases as the iterative search proceeds \cite{soterroni12a,soterroni13}. The process starts from a given $\sigma$, that is decreased by the reduction factor $\beta: 0 <\beta < 1$. When $\sigma$ goes to $0$, $q_{_d}$ tend to unity. The $q$G method starts as a global search method, and gradually becomes a local search, with a similar behavior to the \textit{steepest descent} method \cite{soterroni13}

The step length $\alpha$ is computed by the geometric recursion $\alpha = \beta \cdot \alpha$, where $\beta$ is the same for updating $\sigma$ \cite{soterroni12a,soterroni13}.

The algorithm for the $q$G method is summarized in the \autoref{alg:qg}.

\begin{algorithm}[H]
\caption{$q$-gradient method}
\label{alg:qg}
\footnotesize
\begin{algorithmic}[1]
\State Given initial point $s$, $\sigma>0$, $\alpha>0$ and $0< \beta < 1$
\State $s^b = s$
\While{$N_{FE} < N_{FE}^{qG}$} \Comment{Stopping criterion}
\State Generate $qs$ by a Gaussian distribution with $\sigma$, and $\mu=s$
\State Calculate the $q$-gradient $\nabla_q J(s)$
\State $d = - \nabla_q J(s) / \| \nabla_q J(s) \|$
\State $s = s + \alpha \cdot v$
\If {$J(s) < J(s^b)$}
\State $s^b = s$
\EndIf
\State $\sigma = \beta \cdot \sigma$
\State $\alpha = \beta \cdot \alpha$
\EndWhile
\State \Return $s^b$
\end{algorithmic}
\end{algorithm}

The values of $\sigma$ and $\alpha$ are normalized by the largest distance within the search space $L$, calculated by:
%
\begin{equation}
L=\sqrt{\sum_{d=1}^{D}(s^u_{_d}-s^l_{_d})^2}
\end{equation}

\subsection{Stopping criteria}

The stopping criterion is the maximum number of the function evaluations ($N_{FE}^{qG}$).

\begin{figure}[H]
\caption{Flowchart of the $q$-gradient}
\label{fig:qg}
\centering
\vspace{1em}
\scalebox{.8}{
\includegraphics{images/chapter4_qg.tikz}}
\end{figure}

\section{Hooke-Jeeves Pattern Search Method}
\label{sec:hj}

The direct search method proposed by \citeonline{hooke1961direct} consists of the repeated application of exploratory moves around a base point $s$ defined in a $n$-dimensional space. If these steps are successful, they are followed by pattern moves.

Hooke-Jeeves (HJ) method has been used in some hybrid algorithms, such as HJPCA \cite{RiosCoelho2010843}, HJMA \cite{Moser2009AHB}, PSO/HJ \cite{kampf2010comparison}, GAHJ \cite{long2014hybrid}, AS+HJ \cite{Braun2015}, qG-HJ \cite{HernandezTorres2015}, and MPCA-HJ \cite{HernandezTorres2015b}.

\Autoref{fig:hj,fig:hje}, and \Autoref{alg:hj, alg:hjexploratory} show the flowchart and the pseudo-code of the HJ method, respectively.

\begin{figure}[H]
\caption{Flowchart of the Hooke-Jeeves direct search method}
\label{fig:hj}
\centering
\vspace{1em}
\scalebox{.8}{
\includegraphics{images/chapter4_hjmain.tikz}}
\end{figure}

\begin{figure}[H]
\caption{Flowchart of the Exploratory function in the Hooke-Jeeves direct search method}
\label{fig:hje}
\centering
\vspace{1em}
\scalebox{.8}{
\includegraphics{images/chapter4_hjexp.tikz}}
\end{figure}

\begin{algorithm}[H]
\caption{Hooke-Jeeves pattern search method}
\label{alg:hj}
\begin{algorithmic}[1]
\footnotesize     
\State Choose $s^c, \rho, h_{\mathrm{min}}$
\While{$N_{FE} < N_{FE}^{hj}$ \textbf{and} $h > h_{\mathrm{min}}$}  \Comment{Stopping criteria}
\State $s =$ \Call{Exploratory}{$s^c,h$}
\If{$J(s) < J(s^c)$}
\State $s^{\circ} = s + (s - s^c)$
\If{$J(s^{\circ}) < J(s)$}
\State $s^c = s^{\circ}$
\Else
\State $s^c = s$
\EndIf
\Else
\State $h = h \times \rho$
\EndIf
\EndWhile
\State \Return $s^c$
\end{algorithmic}
\end{algorithm}

\begin{algorithm}[H]
\caption{Exploratory function}
\label{alg:hjexploratory}
\begin{algorithmic}[1]
\footnotesize
\Function{Exploratory}{$s^c,h$}
\State $s = s^c$
\For {$d \gets 1$ to $D$}
\If{$J(s + h v_{_d}) < J(s)$}
\State $s = s + h v_{_d}$
\ElsIf{$J(s - h v_{_d}) < J(s)$}
\State $s = s - h v_{_d}$
\EndIf
\EndFor
\State \Return $s$
\EndFunction
\end{algorithmic}
\end{algorithm}

\subsection{Exploratory move}
In the exploratory move, the solution is changed adding and subtracting each time, a column $v_{_d}$ of the search direction matrix $V$, scaled by a step size $h$.

When $V$ is the identity matrix, the modification is performed on the $d$-th element of the solution $s$ each time. This process is done for all the dimensions of the problem. The original solution is compared with both solutions created, and the best among them will be returned.

\subsection{Pattern move}

A new pattern point $s^{\circ}$ is calculated as follows:
%
\begin{equation}
    s^{\circ} = s + \left(s - s^c \right).
\end{equation}
%
where $s$ is the solution obtained from the exploratory move, and $s^c$ is the current solution. If the pattern point $s^{\circ}$ is better than $s^c$, $s^c$ will be replaced with $s^{\circ}$. If there is no improvement, the step size $h$ is reduced in $\rho$ times.

\subsection{Stopping criteria}

A minimum step size $h_{min}$ and a maximum number of function evaluations $N_{FE}^{hj}$ are defined as stopping criteria.

\subsection{OMPCA-HJ and qG-HJ}

The hybrid algorithms Multi-Particle Collision Algorithm with Hooke-Jeeves (MPCA-HJ) and its variants use an integration scheme of the MPCA and Opposition (OMPCA-HJ), for improving the global search stage. The first phase of exploration is followed by an intensification stage performed by HJ.

Similarly, the $q$-gradient method with Hooke-Jeeves ($q$G-HJ) uses the same approach of the MPCA-HJ, with an exploration phase with $q$G and an exploitation phase with HJ.

\autoref{fig:hybrids} presents the operating flow of both hybrid algorithms.

\begin{figure}[H]%
\caption[Operating flow of the hybrid algorithms]{Operating flow of the hybrid algorithms: \subref{fig:mpcahj} OMPCA-HJ; \subref{fig:qghj} $q$G-HJ;}%
\label{fig:hybrids}%
\vspace{1em}
\centering
\subfigure[][]{%
\label{fig:mpcahj}%
\begin{minipage}{0.45\textwidth}
\centering
\scalebox{0.8}{
\includegraphics{images/chapter4_mpcahj.tikz}}%
\end{minipage}}
\hspace{8pt}%
\subfigure[][]{%
\label{fig:qghj}%
\begin{minipage}{0.45\textwidth}
\centering
\scalebox{0.8}{
\includegraphics{images/chapter4_qghj.tikz}}
\end{minipage}}%
\end{figure}

\section{Chapter conclusions}

